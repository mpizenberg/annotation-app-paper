\section{Presentation of the application}

A screenshot of the application can be seen in Figure~\ref{fig:teaser}.
The image to be annotated occupies the central part of the screen;
a toolbar is located on top, object classes are available on the left
and images to be annotated on the right.


\textbf{Images}.
Multiple images can be loaded at the same time using the image icon
on the top-right corner of the application.
These images are not uploaded on the server,
and can either be loaded locally from the client's machine,
or from a distant server.
% * <matthieu.pizenberg@gmail.com> 2018-05-20T06:53:31.040Z:
% 
% > or from a distant server.
% Seulement quand c'est fait directement dans le champ images des "flags" au démarrage de l'appli, pas en cliquant sur le bouton
% 
% ^ <matthieu.pizenberg@gmail.com> 2018-05-20T09:57:23.955Z.


\textbf{Tools}.
Our application includes several tools to annotate images.
Icons for these tools are depicted in Figure~\ref{fig:icons}.
From left to right, the first available annotation is the point,
that can be useful to designate objects in the image.
It can also be used as a seed in region-growing image segmentation methods.
The second annotation we included is the bounding box,
which provides the localization of objects in the image,
and is used in object detection problems.
The information we acquire are the left, right, top and bottom coordinates
of the bounding box.
The third annotation we chose to implement is the stroke,
or scribble, which is a popular interaction in image segmentation.
It consists in a sequence of points, interpreted as a continuous line.
The outline, fourth type of annotation,
is a closed shape, typically drawn around objects.
It is comparable to a bounding box in essence,
but provides a more precise location of objects.
Finally, polygons can also be drawn (as in LabelMe, for instance),
by successively clicking new points as vertices.


All these tools are available both with a mouse or a touch interaction.
As a matter of fact, some tools are better suited to touch devices
(for example, outlines) than others (polygons).

\begin{figure}[ht]
\centering
\includegraphics[width=0.8\columnwidth]{img/annotation-tools.png}
\caption{Annotation tools icons}%
\label{fig:icons}
\end{figure}


\textbf{Object classes}.
For most annotation tasks, we also need to differentiate objects in the images.
Typically each annotated area is attributed a class, or label.
The PASCAL VOC dataset~\cite{everingham2010pascal}, for example,
is composed of 20 classes, grouped by categories:
\begin{itemize}
\item \textit{Person}: person
\item \textit{Animal}: bird, cat, cow, dog, horse, sheep
\item \textit{Vehicle}: aeroplane, bicycle, boat, bus, car, motorbike, train
\item \textit{Indoor}: bottle, chair, dining table, potted plant, sofa, tv/monitor
\end{itemize}

In our application, classes are specified in a JSON configuration file.
A strict corresponding config for PASCAL VOC classes
is presented in Listing~\ref{lst:pascal}.

\lstinputlisting[language=json,caption={A configuration file to annotate the PASCAL dataset.},float,label={lst:pascal}]{code/config-pascal.json}

To attribute a class to an annotation,
a user should first select the class in the left sidebar,
then use a tool to create an annotation.
Selecting a class in the left sidebar also highlights the annotations
corresponding to this class.


\textbf{Configuration file}.
The five annotation tools are optionally made available by the configuration file.
In Listing~\ref{lst:pascal}, the last line of the depicted configuration file
contains an \texttt{annotations} field, listing the tools that should be available.
In this case, they all are.

In addition to the five fundamental annotation types,
each type can be derived in virtually any number of variations.
For example, interactive segmentation algorithms often require
\textit{foreground} and \textit{background} scribbles.
In our application, this would mean the user would need to draw two types of strokes.
This can be achieved using the configuration file,
as in Listing~\ref{lst:variations}.
Such configuration would result in two stroke icons in the toolbar,
of different colors, just as in Figure~\ref{fig:teaser}.

\lstinputlisting[language=json,caption={A configuration file to include two types of strokes.},float,label={lst:variations}]{code/config-variations.json}

